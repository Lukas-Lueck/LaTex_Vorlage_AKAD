\section{Schlussbetrachtung}
\subsection{Zusammenfassung}
In diesem Assignment wurde die Anwendung von Piezo-Sensoren in der Prozessmesstechnik beschrieben. Dazu wurde zuerst der piezoelektrische Effekt dargestellt. Anschließend erfolgte eine Einführung in die Grundlagen der Strömungsmechanik. Besondere Beachtung fand dabei die Reynolds-Zahl, die K\'{a}rm\'{a}n`sche Wirbelstraße und die Strouhal-Zahl. Mit den erarbeiteten physikalischen Grundlagen konnte die Anwendung von Piezo-Sensoren in der Prozessmesstechnik anhand des Wirbelfrequenzzählers beschrieben werden. Dazu wurde im ersten Schritt der allgemeine Aufbau und die Funktionsweise dieses Messgerätes erarbeitet. Anschließend wurden Wirbelfrequenzmessgeräte der Hersteller $KROHNE$ und \textit{Endress+Hauser} als Beispiel näher beschrieben. Hierbei wurde bereits festgestellt, dass die Messgrößenerfassung nur bei der Firma $KROHNE$ auf einem Piezoelement beruht. Als Abschluss dieses Assignment wurden die beiden Messgeräte gegenübergestellt und markante technische Daten verglichen.
%-------------------------------------------------------------------------------------------------------------------
%-------------------------------------------------------------------------------------------------------------------
\subsection{Kritische Reflexion}
Bei der Literaturrecherche wurde bereits deutlich, dass es eine Vielzahl von physikalischen Prinzipien zur Durchflussmessung gibt. Genauso vielfältig sind auch die Sensoren zur Erfassung der Messgrößen. Dementsprechend müssten die physikalischen Grundlagen auf weitere Gesetzmäßigkeiten ausgeweitet werden. Beispielsweise wird bei dem $Prowirl \: F \: 200$ ein kapazitiver Sensor verwendet. Hier konnte aufgrund des Umfangs des Assignments nur auf Piezo-Sensoren eingegangen werden. Die technischen Eigenschaften der Wirbelfreuqenzzähler wurde nur sehr allgemein betrachtet. Heutige Messgeräte umfassen deutlich mehr Parameter. Mit Blick auf die Industrie 4.0 wird sich in Zukunft der Funktionsumfang von Messgeräten zusätzlich steigern. Auch hier würde eine detaillierte Beschreibung den Umfang des Assingments übersteigen. Abschließend kann festgehalten werden, dass die Ausarbeitung einen guten Einblick in die komplexe Thematik der Anwendung von Piezo-Sensoren in der Prozesstechnik gibt.