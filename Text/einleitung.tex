\section{Einleitung}
Im Zeitalter der Industrie 4.0 spielen Steuerungs- und Automatisierungssysteme eine zentrale Rolle. Bewegungsvorgänge und Zustand der Anlage werden in digital verwertbare Daten umgewandelt und online ausgewertet. Erst diese Informationen machen es möglich, dass Maschinen autonom arbeiten können. Für die Erfassung  und Verarbeitung der Daten gibt es verschiedene technische Möglichkeiten. \autocite[vgl.][24]{Seitz2021} Das Assign\-ment legt den Fokus auf die Verarbeitung der erfassten Informationen mithilfe einer \acp{sps} und dessen Programmierung. \\
Ziel der Arbeit ist es, einen vorher definierten Bewegungsablauf einer Bohreinrichtung in der Programmierumgebung \ac{codesys} zu programmieren. Anschließend soll der Ablauf in einer Simulation dargestellt werden. \\
Im ersten Schritt werden Grundlagen der Steuerungstechnik dargestellt. Hierzu beschreibt der erste Teilabschnitt die für die Aufgabenstellung benötigten Sensoren und Aktoren und die grundlegenden Eigenschaften einer \ac{sps}. Zusätzlich wird die Programmierumgebung \ac{codesys} vorgestellt. Außerdem wird auf das OSI-Schichtmodell eingegangen und welche Ebenen bei einer \ac{sps} Anwendung finden. \\
Anschließend wird der Bewegungsablauf der Bohreinrichtung realisiert. Hierfür ist zuerst das Lastenheft entsprechend der Aufgabenstellung beschrieben. Im Anschluss wird ein Programmablaufplan entworfen. Darauf aufbauend kann die eigentliche Programmierung erstellt werden. Zur Überprüfung der Funktionsfähigkeit des Programms wird der Bewegungsablauf in einer Simulation überprüft. \\
Das Assignment endet mit der Zusammenfassung, in der die Erkenntnisse dieser Arbeit reflektiert werden und eine kritische Auseinandersetzung mit den Ergebnissen erfolgt. 





