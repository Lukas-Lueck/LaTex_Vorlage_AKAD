% Allgemeine Definitionen

% Assignmentbezeichnungen
%-----------------------------------------------------------------------------------
\newcommand*{\assi}{Assignment REG23-AS-002}
% Titel
\newcommand*{\titel}{Steuerungs- und Regelungstechnik} 
% Betreff
\newcommand*{\betreff}{Thema \glqq Bohreinrichtung\grqq{} } 
% Modul
\newcommand*{\modul}{REG23}
% Betreuer
\newcommand*{\dozent}{Heimerdinger} 
% Vor- und Nachname
\newcommand*{\name}{Papa Schlumpf}
% Straße und Hausnummer
\newcommand*{\strasse}{Am Pilz 01} 
% Plz und Ort
\newcommand*{\plzort}{1234 Schlumpfhausen} 
% Immatrikulationsnummer
\newcommand*{\immanr}{123456789}
% Studiengang
\newcommand{\stud}{Mechatronik - Bachelor of \tabto{40mm}Engineering (B. Eng.)}
%-----------------------------------------------------------------------------------

% URL in gleicher Schriftart wie Rest
\urlstyle{same}
% Email mit Verlinkung
\newcommand*{\email}{\href{mailto:test@test.com}{test@test.com}} 

%kein einrücken nach Abbildung/Tabelle
\setlength{\parindent}{0pt}

% Formatierungen für das Literaturverzeichnis
%--------------------------------------------
\xpretobibmacro{author}{\begingroup\bfseries}{}{}
\xapptobibmacro{author}{\endgroup}
% Doppelpunkt und Leerzeichen nach Autor
\renewcommand*{\labelnamepunct}{\addspace in\addcolon\addspace} 
% Punkt am Ende entfernen
\renewcommand*{\finentrypunct}{\addspace}
% Komma zwischen einzelnen Einträgen
\renewcommand*{\newunitpunct}{\addcomma\addspace}
% Punkt am Ende von Zitaten entfernen
\renewcommand*{\bibfootnotewrapper}[1]{\bibsentence#1\addspace}
% Zeilenabstand zwischen einzelnen Einträgen
\setlength{\bibitemsep}{0.5\baselineskip}
%Sortierung bei mehreren Autoren im Format: Name, Vorname
\DeclareNameAlias{sortname}{family-given}
%Trennung zwischen einzelnen Namen mit Semikolon
\renewcommand*{\multinamedelim}{\addsemicolon\space}
\renewcommand*{\finalnamedelim}{\addsemicolon\space}
%Komma zwischen Autor und jahr in Fußnote
\renewcommand*{\nameyeardelim}{\addcomma\space}

%Tabelleneinstellungen:
%Linienstärke
\setlength{\arrayrulewidth}{0.4mm}
%Schriftgröße
\setlength{\tabcolsep}{12pt}
%Spaltendicke
\renewcommand{\arraystretch}{1.1}
%Lininefarbe
%\arrayrulecolor[HTML]{DB5800}


%Einstellung für pdf-Dokumente, die Übernommen werden, auf Überlappung achten!
\includepdfset{width=\paperwidth, height=\paperheight, pagecommand={\thispagestyle{fancy}}}
